\newpage
\newpage
\addcontentsline{toc}{chapter}{ABSTRACT}
\begin{center}
{\large \bf ABSTRACT}
\end{center}
In this project, a compartmental model is developed using infection controlling mechanism
to figure out the advancement of a appropriated attack on severely targeted groups in a computer
network. The model gives an epidemiological design consisting two sub-designs to recognize the
disparity between the global nature of the attacker class and the targeted class. The
targeted nodes are partitioned into four compartments as Susceptible$(S)$, Infected$(I)$, Quarantine$(Q)$ and Recovered$(R)$
whereas the attacker nodes are partitioned into three compartments as Infected$(I)$, Breaking-out$(B)$ and Recovered$(R)$.
The boundedness, the possible feasibility of equilibrium states of the system and their stability are
figured out using cyber mass action incidence phenomenon. Basic reproduction number $R_0$ is
observed for various cases and the results proved that $R_0$ \textless 1 ensures malicious code free stable steady state for the system
and $R_0$$>$1 ensures the existence of endemic
steady state having local asymptotic stability. The impact of controlling
transmission through media coverage controlling coefficient of malicious objects is
analyzed. The infection controlling factor like firewall coefficient '$m$' depends
on the types of files under consideration, defined security rules as per
firewall rule base and the reliability as well as efficiency of the media coverage factor, e.g., firewall. So, on behalf of media coverage factor, it is taken into consideration.
Our objective is to analyze various aspects of malicious code propagation and the effect
of media coverage factor, e.g., firewall security within the computer network.
To achieve the objective, the effect of media coverage factor '$m$' is observed.
The stability of the system is retrieved using local asymptotic stability method.
Finally, Numerical simulation has
been carried out to verify analytic results and most sensitive system parameters for basic reproduction number are
observed using normalized forward sensitivity index. \\

{\it Keywords:} Compartmental model, Computer networks, Local asymptotic stability, Numerical simulation, Basic reproduction number,
Sensitivity analysis, Firewall.
